\documentclass{beamer}
\usetheme{Warsaw}
\usecolortheme{seahorse}
\usepackage{graphicx}
\usepackage{tikz}
\usepackage{listings}
\usepackage{fontawesome5}
\usepackage{xcolor}

% Colors
\definecolor{terminalblack}{RGB}{0,0,0}
\definecolor{terminalgreen}{RGB}{0,200,0}
\definecolor{codegreen}{RGB}{0,128,0}
\definecolor{codepurple}{RGB}{128,0,128}
\definecolor{backcolour}{RGB}{245,245,245}

% Code style for terminal
\lstdefinestyle{terminal}{
    backgroundcolor=\color{terminalblack},
    basicstyle=\ttfamily\small\color{terminalgreen},
    breaklines=true,
    frame=single,
    xleftmargin=0pt,
    framexleftmargin=0pt
}

% Code style for Python
\lstdefinestyle{pythonstyle}{
    backgroundcolor=\color{backcolour},
    commentstyle=\color{codegreen},
    keywordstyle=\color{blue},
    stringstyle=\color{codepurple},
    basicstyle=\ttfamily\small,
    breaklines=true,
    frame=single,
    numbers=left,
    numberstyle=\tiny\color{gray}
}

\title[Elephant Tutorial]{🐘 Elephant}
\subtitle{Complete Guide to Citation Tracking and Boosting}
\author{Malak Alnemari}
\institute{
    \texttt{github.com/alnemari-m/elephant} \\
    \texttt{mohammedalnemari@gmail.com}
}
\date{\today}

\begin{document}

% Title Slide
\begin{frame}
\titlepage
\begin{center}
    \large{\textbf{Never Forget Your Citations}} \\
    \vspace{0.3cm}
    \normalsize{A Complete Tutorial and User Guide}
\end{center}
\end{frame}

% What We'll Cover
\begin{frame}{What You'll Learn Today}
\begin{block}{By the end of this tutorial, you will be able to:}
\begin{enumerate}
    \item Install and set up Elephant on your system
    \item Connect your academic profiles (ORCID, arXiv, etc.)
    \item Fetch and track your citations automatically
    \item Understand your citation metrics and trends
    \item Get personalized recommendations to boost your impact
    \item Export and analyze your citation data
\end{enumerate}
\end{block}

\vspace{0.3cm}

\begin{alertblock}{Prerequisites}
\begin{itemize}
    \item Basic command-line knowledge
    \item Python 3.8+ installed
    \item At least one academic profile (ORCID recommended)
\end{itemize}
\end{alertblock}
\end{frame}

\section{Part 1: Understanding the Problem}

\begin{frame}{Why Do We Need Citation Tracking?}
\begin{columns}
\column{0.5\textwidth}
\textbf{Your Research Journey}
\begin{enumerate}
    \item Publish paper
    \item Wait for citations
    \item Check Google Scholar
    \item Check ORCID
    \item Check arXiv
    \item Manually count citations
    \item Update spreadsheet
    \item Repeat monthly...
\end{enumerate}

\column{0.5\textwidth}
\begin{alertblock}{Problems}
\begin{itemize}
    \item Time-consuming
    \item Error-prone
    \item No historical data
    \item No insights
    \item No actionable advice
\end{itemize}
\end{alertblock}

\vspace{0.3cm}

\begin{exampleblock}{Solution}
Automate everything with Elephant!
\end{exampleblock}
\end{columns}
\end{frame}

\begin{frame}{Real-World Scenario}
\begin{block}{Meet Dr. Sarah}
\begin{itemize}
    \item Published 15 papers over 3 years
    \item Citations scattered across platforms
    \item Spends 2-3 hours/month tracking manually
    \item Doesn't know which papers need promotion
    \item Missing citation opportunities
\end{itemize}
\end{block}

\pause

\begin{exampleblock}{After Using Elephant}
\begin{itemize}
    \item Automated tracking in 5 minutes/month
    \item Discovered 3 under-cited papers
    \item Got recommendations to boost visibility
    \item Increased citations by 40\% in 6 months
    \item Data-driven decisions for paper promotion
\end{itemize}
\end{exampleblock}
\end{frame}

\section{Part 2: Getting Started}

\begin{frame}{What is Elephant?}
\begin{center}
\begin{tikzpicture}[scale=0.9]
% Central elephant
\node[circle, draw, fill=gray!30, minimum size=2.5cm] (elephant) at (0,0) {\Huge 🐘};
\node[below=0.1cm of elephant] {\textbf{Elephant}};

% Surrounding features
\node[rectangle, draw, fill=blue!20, text width=2cm, align=center] (track) at (-3,2) {Track Citations};
\node[rectangle, draw, fill=green!20, text width=2cm, align=center] (analyze) at (3,2) {Analyze Trends};
\node[rectangle, draw, fill=yellow!20, text width=2cm, align=center] (recommend) at (-3,-2) {Get Advice};
\node[rectangle, draw, fill=red!20, text width=2cm, align=center] (boost) at (3,-2) {Boost Impact};

% Arrows
\draw[->, thick] (track) -- (elephant);
\draw[->, thick] (analyze) -- (elephant);
\draw[->, thick] (recommend) -- (elephant);
\draw[->, thick] (boost) -- (elephant);
\end{tikzpicture}
\end{center}

\vspace{0.3cm}

\begin{block}{In Simple Terms}
Elephant is your \textbf{personal citation assistant} that remembers everything about your research impact and helps you improve it.
\end{block}
\end{frame}

\begin{frame}{How Elephant Collects Your Data}
\begin{center}
\begin{tikzpicture}[scale=0.85, every node/.style={scale=0.85}]
% Platforms at top
\node[draw, rectangle, fill=blue!30, minimum width=1.5cm] (orcid) at (0,5) {ORCID};
\node[draw, rectangle, fill=blue!30, minimum width=1.5cm] (arxiv) at (2,5) {arXiv};
\node[draw, rectangle, fill=blue!30, minimum width=1.5cm] (sem) at (4,5) {Semantic};
\node[draw, rectangle, fill=blue!30, minimum width=1.5cm] (scholar) at (6,5) {Scholar};

% APIs
\node[draw, rectangle, fill=green!30, minimum width=7cm, minimum height=0.7cm] (api) at (3,3.5) {API Connectors};

% Elephant core
\node[circle, fill=gray!30, minimum size=1.5cm] (core) at (3,2) {🐘};

% Database
\node[draw, cylinder, fill=yellow!30, minimum width=2cm, minimum height=1cm, shape aspect=0.3] (db) at (3,0.3) {Your Data};

% You
\node[draw, circle, fill=purple!30, minimum size=1.2cm] (you) at (3,-1.5) {You};

% Arrows
\draw[->, thick] (orcid) -- (api);
\draw[->, thick] (arxiv) -- (api);
\draw[->, thick] (sem) -- (api);
\draw[->, thick] (scholar) -- (api);
\draw[->, thick] (api) -- (core);
\draw[->, thick] (core) -- (db);
\draw[<->, thick] (you) -- (db);

% Labels
\node[right=0.1cm of orcid] {\tiny Publications};
\node[right=0.1cm of sem] {\tiny Citations};
\end{tikzpicture}
\end{center}
\end{frame}

\begin{frame}[fragile]{Installation - Step by Step}
\begin{block}{Step 1: Open Terminal}
On Linux/Mac: Open Terminal \\
On Windows: Open PowerShell or Git Bash
\end{block}

\begin{block}{Step 2: Clone Repository}
\begin{lstlisting}[style=terminal]
$ git clone https://github.com/alnemari-m/elephant
$ cd elephant
\end{lstlisting}
\end{block}

\begin{block}{Step 3: Install Dependencies}
\begin{lstlisting}[style=terminal]
$ pip install -r requirements.txt
\end{lstlisting}
\textit{This installs all required libraries}
\end{block}

\begin{block}{Step 4: Install Elephant}
\begin{lstlisting}[style=terminal]
$ pip install -e .
\end{lstlisting}
\textit{Now you can use the \texttt{elephant} command anywhere!}
\end{block}
\end{frame}

\begin{frame}[fragile]{First Time Setup}
\begin{block}{Run the Initialize Command}
\begin{lstlisting}[style=terminal]
$ elephant init
\end{lstlisting}
\end{block}

\pause

\begin{block}{You'll Be Asked For:}
\begin{lstlisting}[style=terminal]
Your ORCID ID: 0000-0002-1234-5678
Your email: sarah@university.edu
Your name: Dr. Sarah Johnson
\end{lstlisting}
\end{block}

\pause

\begin{exampleblock}{What Just Happened?}
\begin{itemize}
    \item Created config folder: \texttt{\textasciitilde/.elephant/}
    \item Saved your settings: \texttt{config.yaml}
    \item Initialized database: \texttt{citations.db}
\end{itemize}
\end{exampleblock}
\end{frame}

\begin{frame}[fragile]{Finding Your ORCID ID}
\begin{block}{What is ORCID?}
ORCID is a unique identifier for researchers (like a DOI for people!)
\end{block}

\begin{enumerate}
    \item Go to \texttt{https://orcid.org}
    \item Search for your name
    \item Your ORCID looks like: \texttt{0000-0002-1234-5678}
\end{enumerate}

\vspace{0.3cm}

\begin{alertblock}{Don't Have an ORCID?}
\begin{enumerate}
    \item Go to \texttt{https://orcid.org/register}
    \item Create free account (takes 2 minutes)
    \item Add your publications
    \item Get your ORCID ID
\end{enumerate}
\end{alertblock}

\begin{center}
\textbf{ORCID is essential for modern academic research!}
\end{center}
\end{frame}

\section{Part 3: Using Elephant}

\begin{frame}[fragile]{Your First Data Fetch}
\begin{block}{Fetch Data from All Platforms}
\begin{lstlisting}[style=terminal]
$ elephant fetch --all
\end{lstlisting}
\end{block}

\pause

\begin{exampleblock}{What You'll See}
\begin{lstlisting}[style=terminal, basicstyle=\ttfamily\tiny\color{terminalgreen}]
Fetching citation data...

Fetching from orcid...
[OK] orcid: 15 papers, 0 citations

Fetching from arxiv...
[OK] arxiv: 8 papers, 0 citations

Fetching from semantic_scholar...
[OK] semantic_scholar: 15 papers, 387 citations

Data fetch completed!
\end{lstlisting}
\end{exampleblock}

\begin{alertblock}{Note}
First fetch may take 1-2 minutes depending on your publication count
\end{alertblock}
\end{frame}

\begin{frame}{Understanding Platform Roles}
\begin{center}
\begin{tabular}{|l|l|l|}
\hline
\textbf{Platform} & \textbf{Provides} & \textbf{Best For} \\
\hline
ORCID & Publications & Your paper list \\
arXiv & Preprints & CS/Physics papers \\
Semantic Scholar & Citations & Citation counts \\
Google Scholar & Everything & Complete coverage \\
\hline
\end{tabular}
\end{center}

\vspace{0.5cm}

\begin{block}{Recommendation}
\begin{itemize}
    \item \textbf{Always use:} ORCID + Semantic Scholar
    \item \textbf{Optional:} arXiv (if you post preprints)
    \item \textbf{Careful with:} Google Scholar (can be blocked)
\end{itemize}
\end{block}
\end{frame}

\begin{frame}[fragile]{Viewing Your Dashboard}
\begin{block}{Basic Dashboard}
\begin{lstlisting}[style=terminal]
$ elephant dashboard
\end{lstlisting}
\end{block}

\begin{block}{Detailed Dashboard with Top Papers}
\begin{lstlisting}[style=terminal]
$ elephant dashboard --detailed
\end{lstlisting}
\end{block}

\begin{block}{Dashboard for Specific Period}
\begin{lstlisting}[style=terminal]
$ elephant dashboard --period week
$ elephant dashboard --period month
$ elephant dashboard --period year
\end{lstlisting}
\end{block}
\end{frame}

\begin{frame}{Dashboard Explained}
\begin{block}{Citation Dashboard for Dr. Sarah Johnson}
\texttt{ORCID: 0000-0002-1234-5678} \\
\texttt{Period: All time}
\end{block}

\begin{block}{Overall Metrics}
\begin{tabular}{lrr}
\textbf{Metric} & \textbf{Value} & \textbf{Change} \\
\hline
Total Papers & 15 & +2 \\
Total Citations & 387 & +23 \\
H-Index & 12 & +1 \\
Avg Citations/Paper & 25.8 & --- \\
\end{tabular}
\end{block}

\begin{alertblock}{What Does This Mean?}
\begin{itemize}
    \item \textbf{H-Index 12}: You have 12 papers with at least 12 citations each
    \item \textbf{+23 citations}: 23 new citations since last fetch
    \item \textbf{Avg 25.8}: Your papers average 26 citations
\end{itemize}
\end{alertblock}
\end{frame}

\begin{frame}{Understanding H-Index}
\begin{block}{What is H-Index?}
A metric that balances productivity and citation impact
\end{block}

\vspace{0.3cm}

\begin{exampleblock}{Example: H-Index = 5}
You have \textbf{5 papers} with \textbf{at least 5 citations each}

\begin{center}
\begin{tabular}{lc}
Paper & Citations \\
\hline
Paper A & 87 ← counts \\
Paper B & 56 ← counts \\
Paper C & 12 ← counts \\
Paper D & 8 ← counts \\
Paper E & 5 ← counts \\
Paper F & 3 ← doesn't count \\
Paper G & 1 ← doesn't count \\
\end{tabular}
\end{center}
\end{exampleblock}

\begin{block}{Why It Matters}
Higher H-index → Consistent high-impact research \\
Used for grants, promotions, job applications
\end{block}
\end{frame}

\begin{frame}[fragile]{Getting Recommendations}
\begin{block}{Get Top 5 Suggestions}
\begin{lstlisting}[style=terminal]
$ elephant recommend
\end{lstlisting}
\end{block}

\begin{block}{Get More Recommendations}
\begin{lstlisting}[style=terminal]
$ elephant recommend --top 10
\end{lstlisting}
\end{block}

\begin{block}{Filter by Category}
\begin{lstlisting}[style=terminal]
$ elephant recommend --category visibility
$ elephant recommend --category collaboration
$ elephant recommend --category trending
\end{lstlisting}
\end{block}
\end{frame}

\begin{frame}{Example Recommendation}
\begin{block}{Recommendation 1 - Visibility}
\textbf{Promote under-cited paper}

\vspace{0.2cm}

Your paper "Deep Learning for Medical Imaging" has only 3 citations after 2 years. Consider promoting it.

\vspace{0.3cm}

\textbf{Impact:} High - Can significantly increase citations \\
\textbf{Effort:} Medium - Requires targeted outreach

\vspace{0.3cm}

\textbf{Action:} Share on ResearchGate, Academia.edu, and relevant social media. Contact researchers in the field. Present at conferences.
\end{block}

\begin{alertblock}{Why This Recommendation?}
\begin{itemize}
    \item Paper is 2 years old (past early citation period)
    \item Only 3 citations (below average for your profile)
    \item System detected potential for higher impact
\end{itemize}
\end{alertblock}
\end{frame}

\begin{frame}{Example Recommendation 2}
\begin{block}{Recommendation 2 - Collaboration}
\textbf{Increase collaboration}

\vspace{0.2cm}

7 papers have minimal co-authors. Collaborative papers typically receive more citations.

\vspace{0.3cm}

\textbf{Impact:} High - Collaborative papers get 2-3x more citations \\
\textbf{Effort:} High - Requires networking

\vspace{0.3cm}

\textbf{Action:} Reach out to researchers in your field. Join research groups. Attend conferences and workshops. Use platforms like ResearchGate to connect.
\end{block}

\begin{exampleblock}{Research Shows}
Papers with 3+ authors get \textbf{2.5x more citations} than solo papers!
\end{exampleblock}
\end{frame}

\begin{frame}[fragile]{Tracking Specific Papers}
\begin{block}{Track by DOI}
\begin{lstlisting}[style=terminal]
$ elephant track --doi "10.1234/example.2023"
\end{lstlisting}
\end{block}

\begin{block}{Track by arXiv ID}
\begin{lstlisting}[style=terminal]
$ elephant track --arxiv "2301.12345"
\end{lstlisting}
\end{block}

\begin{block}{Track by Title}
\begin{lstlisting}[style=terminal]
$ elephant track --title "Deep Learning"
\end{lstlisting}
\end{block}

\begin{block}{List All Tracked Papers}
\begin{lstlisting}[style=terminal]
$ elephant track --list
\end{lstlisting}
\end{block}

\begin{alertblock}{Why Track Papers?}
Monitor your most important papers more closely!
\end{alertblock}
\end{frame}

\begin{frame}[fragile]{Exporting Your Data}
\begin{block}{Export to CSV (Default)}
\begin{lstlisting}[style=terminal]
$ elephant export
\end{lstlisting}
\end{block}

\begin{block}{Export to JSON}
\begin{lstlisting}[style=terminal]
$ elephant export --format json --output my_data.json
\end{lstlisting}
\end{block}

\begin{block}{Export to Excel}
\begin{lstlisting}[style=terminal]
$ elephant export --format xlsx --output citations.xlsx
\end{lstlisting}
\end{block}

\begin{exampleblock}{Use Cases}
\begin{itemize}
    \item Create visualizations in Excel/R/Python
    \item Include in grant applications
    \item Share with collaborators
    \item Build custom reports
\end{itemize}
\end{exampleblock}
\end{frame}

\section{Part 4: Advanced Features}

\begin{frame}{Setting Up API Keys (Optional)}
\begin{block}{Why Add API Keys?}
\begin{itemize}
    \item Faster data fetching
    \item Access to more data
    \item Fewer rate limits
    \item Better reliability
\end{itemize}
\end{block}

\begin{alertblock}{Which APIs Should You Get?}
\begin{enumerate}
    \item \textbf{Semantic Scholar} - FREE, highly recommended!
    \item \textbf{ORCID} - FREE, for private data access
    \item \textbf{Web of Science / Scopus} - Requires institution
\end{enumerate}
\end{alertblock}
\end{frame}

\begin{frame}[fragile]{Getting Semantic Scholar API Key}
\begin{enumerate}
    \item Go to \texttt{https://www.semanticscholar.org/product/api}
    \item Click "Request API Key"
    \item Fill out form (takes 2 minutes)
    \item Receive key via email (instant to 24 hours)
\end{enumerate}

\vspace{0.5cm}

\begin{block}{Add to Elephant}
\begin{lstlisting}[style=terminal]
$ cd ~/.elephant
$ nano .env
\end{lstlisting}

Add this line:
\begin{lstlisting}[style=terminal]
SEMANTIC_SCHOLAR_API_KEY=your_key_here
\end{lstlisting}

Save and exit (Ctrl+X, Y, Enter)
\end{block}

\begin{exampleblock}{Benefits}
\begin{itemize}
    \item 10x more API calls per day
    \item Faster response times
    \item Access to more detailed data
\end{itemize}
\end{exampleblock}
\end{frame}

\begin{frame}[fragile]{Setting Up Citation Alerts}
\begin{block}{Enable Alerts}
\begin{lstlisting}[style=terminal]
$ elephant alert --enable
\end{lstlisting}
\end{block}

\begin{block}{Set Alert Threshold}
\begin{lstlisting}[style=terminal]
$ elephant alert --threshold 5
\end{lstlisting}
\textit{Get alerted when a paper reaches 5+ citations}
\end{block}

\begin{block}{Disable Alerts}
\begin{lstlisting}[style=terminal]
$ elephant alert --disable
\end{lstlisting}
\end{block}

\begin{exampleblock}{Future Feature}
Email notifications coming soon! \\
For now, alerts are stored in the database
\end{exampleblock}
\end{frame}

\begin{frame}[fragile]{Detailed Paper Statistics}
\begin{block}{Get Stats for Specific Paper}
\begin{lstlisting}[style=terminal]
$ elephant stats --paper "10.1234/example"
\end{lstlisting}
\end{block}

\begin{block}{Output Example}
\textbf{Paper Statistics} \\
\rule{\textwidth}{0.4pt}

\textbf{Deep Learning for Medical Image Analysis}

Citations: 87 \\
Year: 2020 \\
Venue: Nature Medicine \\
DOI: 10.1234/example

\textbf{Citation Growth:}
\begin{itemize}
    \item Last 7 days: +2
    \item Last 30 days: +7
    \item Last year: +34
\end{itemize}
\end{block}
\end{frame}

\section{Part 5: Practical Workflows}

\begin{frame}{Weekly Workflow}
\begin{block}{Every Monday Morning (5 minutes)}
\begin{enumerate}
    \item Fetch latest data
    \begin{itemize}
        \item \texttt{elephant fetch --all}
    \end{itemize}

    \item Check dashboard
    \begin{itemize}
        \item \texttt{elephant dashboard --detailed}
    \end{itemize}

    \item Review recommendations
    \begin{itemize}
        \item \texttt{elephant recommend}
    \end{itemize}

    \item Act on top recommendation
    \begin{itemize}
        \item Share paper, email researcher, update profile, etc.
    \end{itemize}
\end{enumerate}
\end{block}

\begin{exampleblock}{Result}
Stay informed, act strategically, boost citations systematically!
\end{exampleblock}
\end{frame}

\begin{frame}{Monthly Review Workflow}
\begin{block}{First of Every Month (15 minutes)}
\begin{enumerate}
    \item Fetch data: \texttt{elephant fetch --all}

    \item Review monthly trends: \texttt{elephant dashboard --period month}

    \item Export data: \texttt{elephant export --format xlsx}

    \item Analyze in Excel/Sheets
    \begin{itemize}
        \item Which papers growing fastest?
        \item Which papers stagnating?
        \item Overall citation velocity?
    \end{itemize}

    \item Plan promotion strategy
    \begin{itemize}
        \item Based on recommendations
        \item Focus on under-cited papers
        \item Identify collaboration opportunities
    \end{itemize}
\end{enumerate}
\end{block}
\end{frame}

\begin{frame}{Grant Application Workflow}
\begin{block}{When Applying for Grants}
\begin{enumerate}
    \item Fetch latest data
    \begin{itemize}
        \item \texttt{elephant fetch --all}
    \end{itemize}

    \item Export comprehensive report
    \begin{itemize}
        \item \texttt{elephant export --format xlsx}
    \end{itemize}

    \item Include in application:
    \begin{itemize}
        \item Total citations
        \item H-index
        \item Citation trends (growth chart)
        \item Top-cited papers
    \end{itemize}

    \item Show impact metrics
    \begin{itemize}
        \item Papers in high-impact journals
        \item Collaboration breadth
        \item Field influence
    \end{itemize}
\end{enumerate}
\end{block}
\end{frame}

\section{Part 6: Troubleshooting}

\begin{frame}[fragile]{Common Issues and Solutions}
\begin{alertblock}{Issue 1: "Not initialized" Error}
\textbf{Solution:}
\begin{lstlisting}[style=terminal]
$ elephant init
\end{lstlisting}
You need to initialize first!
\end{alertblock}

\begin{alertblock}{Issue 2: No Citations Fetched}
\textbf{Possible Causes:}
\begin{itemize}
    \item Papers not yet indexed (wait 1-2 weeks after publication)
    \item ORCID ID incorrect
    \item Papers missing DOIs
\end{itemize}

\textbf{Solution:}
\begin{itemize}
    \item Verify ORCID at \texttt{orcid.org}
    \item Add DOIs to papers
    \item Try fetching from Semantic Scholar specifically
\end{itemize}
\end{alertblock}
\end{frame}

\begin{frame}[fragile]{More Troubleshooting}
\begin{alertblock}{Issue 3: Google Scholar Blocked}
\textbf{Error:} HTTP 429 or similar

\textbf{Solution:}
\begin{itemize}
    \item Google Scholar blocks automated requests
    \item Disable it: Edit \texttt{\textasciitilde/.elephant/config.yaml}
    \item Set \texttt{google\_scholar: enabled: false}
    \item Use Semantic Scholar instead (better API!)
\end{itemize}
\end{alertblock}

\begin{alertblock}{Issue 4: Missing Papers}
\textbf{Solution:}
\begin{itemize}
    \item Check papers are in ORCID profile
    \item Ensure DOIs are correct
    \item Some preprints may not be indexed yet
    \item Manually track important papers: \texttt{elephant track --doi "..."}
\end{itemize}
\end{alertblock}
\end{frame}

\section{Part 7: Best Practices}

\begin{frame}{Best Practices for Citation Tracking}
\begin{block}{Do's ✓}
\begin{itemize}
    \item Update ORCID regularly with new papers
    \item Fetch data weekly for consistent tracking
    \item Act on recommendations systematically
    \item Export data monthly for trend analysis
    \item Add DOIs to all publications
    \item Use Semantic Scholar API (free!)
\end{itemize}
\end{block}

\begin{alertblock}{Don'ts ✗}
\begin{itemize}
    \item Don't fetch more than once per day (unnecessary)
    \item Don't rely only on Google Scholar (can be blocked)
    \item Don't ignore low-cited papers (they need attention!)
    \item Don't forget to backup exported data
\end{itemize}
\end{alertblock}
\end{frame}

\begin{frame}{Maximizing Your Citations}
\begin{block}{Quick Wins (This Week)}
\begin{enumerate}
    \item Complete all online profiles (ORCID, Google Scholar, ResearchGate)
    \item Add missing DOIs to papers
    \item Share your top 3 papers on social media
\end{enumerate}
\end{block}

\begin{block}{Medium-term (1-3 Months)}
\begin{enumerate}
    \item Post preprints to arXiv/bioRxiv
    \item Present at conferences
    \item Email 10 researchers about your work
    \item Write a blog post explaining your research
\end{enumerate}
\end{block}

\begin{block}{Long-term (6-12 Months)}
\begin{enumerate}
    \item Write a review paper (gets 3-5x more citations!)
    \item Increase collaboration
    \item Media outreach for high-impact work
\end{enumerate}
\end{block}
\end{frame}

\begin{frame}{Understanding Your Metrics}
\begin{center}
\begin{tabular}{|l|l|l|}
\hline
\textbf{Metric} & \textbf{Good} & \textbf{Great} \\
\hline
H-Index (5 years) & 10+ & 20+ \\
Citations/Paper & 15+ & 30+ \\
Total Citations & 200+ & 1000+ \\
Papers/Year & 2-3 & 4-5 \\
\hline
\end{tabular}
\end{center}

\vspace{0.5cm}

\begin{alertblock}{Important Note}
Metrics vary GREATLY by field! \\
Compare yourself to researchers in YOUR field, not across all fields.

\vspace{0.3cm}

\begin{itemize}
    \item Biology: Higher citations
    \item Mathematics: Lower citations
    \item Computer Science: Medium citations
\end{itemize}
\end{alertblock}
\end{frame}

\section{Part 8: Real Examples}

\begin{frame}{Example 1: Promoting Under-Cited Paper}
\begin{block}{Situation}
Paper published 18 months ago, only 2 citations \\
Elephant recommendation: "Promote under-cited paper"
\end{block}

\begin{exampleblock}{Actions Taken}
\begin{enumerate}
    \item Posted on ResearchGate (day 1)
    \item Shared on Twitter with plain-language summary (day 1)
    \item Emailed 5 researchers citing similar work (day 2)
    \item Submitted to relevant conference (week 2)
    \item Wrote blog post explaining findings (week 3)
\end{enumerate}
\end{exampleblock}

\begin{block}{Results (6 months)}
\begin{itemize}
    \item Citations increased from 2 to 15
    \item New collaboration started from email outreach
    \item Paper accepted at conference
    \item Blog post viewed 500+ times
\end{itemize}
\end{block}
\end{frame}

\begin{frame}{Example 2: Systematic Weekly Tracking}
\begin{block}{Researcher Profile}
PhD student, 5 papers published, struggling to track impact
\end{block}

\begin{exampleblock}{Implemented Workflow}
\textbf{Every Monday 9am:}
\begin{enumerate}
    \item \texttt{elephant fetch --all}
    \item \texttt{elephant dashboard --detailed}
    \item Review weekly changes
    \item Act on \#1 recommendation
\end{enumerate}

\textbf{Time investment:} 5 minutes/week
\end{exampleblock}

\begin{block}{Results (3 months)}
\begin{itemize}
    \item Discovered 2 new citations immediately
    \item Identified collaboration opportunity
    \item Added missing DOIs (+visibility)
    \item Started tracking citation velocity
    \item More confident about research impact
\end{itemize}
\end{block}
\end{frame}

\begin{frame}{Example 3: Preparing for Job Applications}
\begin{block}{Situation}
Postdoc applying for faculty positions, needs strong impact metrics
\end{block}

\begin{exampleblock}{Using Elephant}
\begin{enumerate}
    \item Exported complete citation history
    \item Created Excel charts showing citation growth
    \item Calculated field-normalized metrics
    \item Identified most impactful papers (for talks)
    \item Tracked h-index improvement over time
\end{enumerate}
\end{exampleblock}

\begin{block}{Application Materials}
\begin{itemize}
    \item CV: "387 citations, h-index 12, growing 35\%/year"
    \item Research statement: Included citation growth charts
    \item Job talk: Featured most-cited work
    \item Interview prep: Confident about impact metrics
\end{itemize}
\end{block}

\begin{exampleblock}{Outcome}
Received 3 job offers! Impact metrics were conversation topic in all interviews.
\end{exampleblock}
\end{frame}

\section{Part 9: Q\&A and Resources}

\begin{frame}{Frequently Asked Questions}
\begin{block}{Q: How often should I fetch data?}
\textbf{A:} Once per week is ideal. More frequent fetching doesn't help (citations don't change daily).
\end{block}

\begin{block}{Q: Which platforms should I enable?}
\textbf{A:} ORCID + Semantic Scholar are essential. arXiv if you post preprints. Skip Google Scholar unless needed (gets blocked).
\end{block}

\begin{block}{Q: Is my data private?}
\textbf{A:} Yes! Everything stored locally on your computer in \texttt{\textasciitilde/.elephant/}. Nothing sent to external servers except API calls to public platforms.
\end{block}

\begin{block}{Q: Can I use this for my research group?}
\textbf{A:} Absolutely! Each member can track their own citations, or combine data for group metrics.
\end{block}
\end{frame}

\begin{frame}{More FAQ}
\begin{block}{Q: What if my field has low citation rates?}
\textbf{A:} Elephant shows relative metrics. Compare to peers in YOUR field, not across all fields. Use recommendations to improve within your context.
\end{block}

\begin{block}{Q: Can I import historical data?}
\textbf{A:} Currently no, but you can export from other tools and import to database manually. Future feature planned!
\end{block}

\begin{block}{Q: How do I back up my data?}
\textbf{A:}
\begin{enumerate}
    \item Copy \texttt{\textasciitilde/.elephant/} folder
    \item Or export regularly: \texttt{elephant export}
\end{enumerate}
\end{block}

\begin{block}{Q: Can I contribute to Elephant?}
\textbf{A:} Yes! It's open source. Check \texttt{github.com/alnemari-m/elephant}
\end{block}
\end{frame}

\begin{frame}{Resources and Documentation}
\begin{block}{Elephant Resources}
\begin{itemize}
    \item \textbf{GitHub:} \texttt{github.com/alnemari-m/elephant}
    \item \textbf{Usage Guide:} \texttt{USAGE\_GUIDE.md}
    \item \textbf{Citation Strategies:} \texttt{CITATION\_BOOST\_STRATEGIES.md}
    \item \textbf{Issues/Support:} GitHub Issues
\end{itemize}
\end{block}

\begin{block}{Platform Resources}
\begin{itemize}
    \item \textbf{ORCID:} \texttt{orcid.org}
    \item \textbf{Semantic Scholar:} \texttt{semanticscholar.org}
    \item \textbf{arXiv:} \texttt{arxiv.org}
    \item \textbf{Google Scholar:} \texttt{scholar.google.com}
\end{itemize}
\end{block}

\begin{block}{Learning More}
\begin{itemize}
    \item See \texttt{CITATION\_BOOST\_STRATEGIES.md} for 20+ proven strategies
    \item Join \#AcademicTwitter for citation tips
    \item Read blog posts about research impact
\end{itemize}
\end{block}
\end{frame}

\begin{frame}{Summary: What You Learned}
\begin{block}{You now know how to:}
\begin{enumerate}
    \item Install and set up Elephant
    \item Connect your academic profiles
    \item Fetch citation data automatically
    \item Read and understand your metrics
    \item Get actionable recommendations
    \item Track specific papers
    \item Export and analyze data
    \item Create effective workflows
    \item Troubleshoot common issues
    \item Maximize your citation impact
\end{enumerate}
\end{block}

\begin{exampleblock}{Next Steps}
\begin{enumerate}
    \item Install Elephant today
    \item Set up your first fetch
    \item Review your dashboard
    \item Act on one recommendation this week
\end{enumerate}
\end{exampleblock}
\end{frame}

\begin{frame}
\begin{center}
\Huge{Thank You!}

\vspace{1cm}

\Large{Never Forget Your Citations}

\vspace{1cm}

\begin{tikzpicture}
\node at (0,0) {\Huge 🐘};
\end{tikzpicture}

\vspace{1cm}

\normalsize{
\textbf{Questions?} \\
\vspace{0.3cm}
\texttt{github.com/alnemari-m/elephant} \\
\texttt{mohammedalnemari@gmail.com}
}
\end{center}
\end{frame}

\begin{frame}{Quick Reference Card}
\begin{block}{Essential Commands}
\begin{description}
    \item[\texttt{elephant init}] First-time setup
    \item[\texttt{elephant fetch --all}] Get latest data
    \item[\texttt{elephant dashboard}] View metrics
    \item[\texttt{elephant recommend}] Get advice
    \item[\texttt{elephant track}] Monitor paper
    \item[\texttt{elephant export}] Save data
\end{description}
\end{block}

\begin{block}{Files \& Folders}
\begin{description}
    \item[\texttt{\textasciitilde/.elephant/config.yaml}] Your settings
    \item[\texttt{\textasciitilde/.elephant/.env}] API keys
    \item[\texttt{\textasciitilde/.elephant/data/}] Your database
\end{description}
\end{block}

\vspace{0.3cm}
\begin{center}
\small{Bookmark this slide for quick reference!}
\end{center}
\end{frame}

\end{document}
